\section{first antiphon}

\begin{rubricmed}
    On Sundays is sung the first typical antiphon:
\end{rubricmed}

\begin{liturgicaltext}
    \choir Bless the Lord, O my Soul; blessed art Thou, O Lord. Bless the Lord, O my soul, and all that is within me bless His holy name. Bless the Lord, O my soul, and forget not all His benefits. Who forgives all thine iniquities, Who heals all thy diseases. The Lord is compassionate and merciful, longsuffering and of great goodness. Bless the Lord, O my soul: blessed art Thou, O Lord.
\end{liturgicaltext}

\begin{rubricmed}
    But on weekdays is sung the first daily antiphon:
\end{rubricmed}

\begin{liturgicaltext}
    \choir It is good to give praise unto the Lord, and to chant unto Thy name, O Most High.
    \item[] \textit{Refrain:} Through the prayers of the Theotokos, O Savior, save us.
    \item[] To proclaim in the morning Thy mercy, and Thy truth by night. \textit{(refrain)}
    \item[] That upright is the Lord, and there is no unrighteousness in Him. \textit{(refrain)}
    \item[] Glory to the Father, and to the Son, and to the Holy Spirit, both now and ever, and unto the ages of ages. Amen.
\end{liturgicaltext}

\begin{rubricmed}
    On feast days are sung the festal antiphons, which share their melody with the daily antiphons.
\end{rubricmed}